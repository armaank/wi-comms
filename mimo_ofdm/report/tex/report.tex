\documentclass[conference]{IEEEtran}

\usepackage{amsmath}
\usepackage{graphicx}
%\usepackage[backend=bibtex,style=chem-rsc]{biblatex}
\usepackage{lettrine}
\usepackage{cite}
\usepackage{float}
\usepackage{blindtext}
\usepackage{eso-pic}
\usepackage[utf8]{inputenc}
\usepackage[english]{babel}
\usepackage[numbers]{natbib}
\usepackage{hyperref}
\usepackage{booktabs}
\usepackage{filecontents}
\newcommand\tab[1][1cm]{\hspace*{#1}}
\newcommand\AtPageUpperMyright[1]{\AtPageUpperLeft{%
    \put(\LenToUnit{0.5\paperwidth},\LenToUnit{-1cm}){%
     \parbox{0.5\textwidth}{\raggedleft\fontsize{9}{11}\selectfont #1}}%
    }}%
    \newcommand{\conf}[1]{%
    \AddToShipoutPictureBG*{%
    \AtPageUpperMyright{#1}}
}

\title{Sample paper}

\author{
  \IEEEauthorblockN{Armaan Kohli}
  \IEEEauthorblockA{\textit{Department of Electrical Engineering} \\
\textit{The Cooper Union for the Advancement of Science and Art}\\
New York City, United States \\
kohli@cooper.edu\\
\href{github.com/armaank/wi-comms}{https://github.com/armaank/wi-comms}}}

\begin{document}
\title{MIMI-OFDM Wireless Link Simulations}

\maketitle
\conf{ECE-408: WIRELESS COMMUNICATIONS, April 2020}

\begin{abstract}
 We simulate a MIMO wireless link and implement a simple OFDM scheme based on the IEEE802.11a standard with several different types of equalization techniques. We then design a MIMO-OFDM system and conduct a performance analysis of the wireless link. 
\end{abstract}

\begin{IEEEkeywords}
MIMO, OFDM, MIMO-OFDM, IEEE802.11a, channel estimation, zero-forcing equalization, minimum mean-squared error equalization.

\end{IEEEkeywords}

\section{Introduction}
\lettrine[findent=2pt]{\textbf{M}}{ }IMO-OFDM is a technique used in large scale wireless systems. MIMO, multi-input mutli-output, in the context of communications, means that the system consists of multiple transmitters and multiple receivers. OFDM, orthogonal frequency division multiplexing, is a technique used to 

\section{MIMO Link}

\subsection{Channel Precoding}
The first method for channel equalization is called 

\subsection{ZF Equalization}

\subsection{MMSE Equalization}

\subsection{Results}

\section{OFDM LINK}

\section{802.11a PHY Layer Overview}

\section{ZF Equalization}

\subsection{MMSE Equalization}


\subsection{Results}

\section{Hybrid MIMO-OFDM System}

\subsection{Results}

\section{Conclusion} 
We successfully simulated a MIMO wireless link and an OFDM scheme based on the IEEE802.11a PHY layer standard, experimenting with different methods for equalization and performance enhancement. We then implimented a hybrid MIMO-OFDM system and compared the performance of various equalization schemes.  

\bibliographystyle{unsrt}
\bibliography{bib}
\end{document}